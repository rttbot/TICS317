\chapterimage{Pictures/tricycle-691587_1920.jpg}
\chapter{Propuesta de Proyecto de asignatura Arquitectura de Sistemas}
\vspace{80px}
\begin{flushright}
\textit{Ver charla TEDx: El futuro de las ciudades inteligentes \url{https://www.youtube.com/watch?v=8siOG-WU5ZQ})del Doctor Arjan van Timmeren}.
\end{flushright}


La ley 21.180 ``Transformación digital del Estado'' establece la obligatoriedad del soporte electrónico, de manera que todos los nuevos trámites y servicios que el Estado ofrece a los ciudadanos sean preferentemente digitales. En ese sentido es importante denotar que Transformación digital no es sólo digitalización. Para facilitar la transformación digital, debemos generar un espacio donde la tetra-hélice compuesta por personas, entidades públicas, privadas y centros de investigación puedan co-crear los nuevos procesos para alcanzar los valores púbicos  a nivel de  país con la premisa de que todo está disponible digitalmente en un ambiente seguro. 
%
Es importante considerar el índice de ciudad inteligente \url{https://www.imd.org/research-knowledge/reports/imd-smart-city-index-2019/} el cual se enfoca principalmente en cómo los ciudadanos perciben el alcance y el impacto de los esfuerzos que hacen las ciudades más ``inteligentes'', balanceando los aspectos económicos y tecnológicos con dimensiones humanas. 


\section{Trabajo de Investigación: Propuesta de proyecto de transformación digital/smart city para mi país}
Ustedes son parte de una institución pública del estado. Pertenecen a la Unidad de Innovación, donde sus tareas comunes son realizar investigación que permita realizar propuestas para resolver problemáticas de nuevas maneras, solicitando el desarrollo de prototipos que le permitan validar sus hipótesis y validar su teoría en la realidad. Asumiendo que se ha solicitado a todos el realizar una bajada de lo que significa la trasnformación digital y ser capaces de detectar necesidades no completamente cubiertas desde el punto de vista de un país sostenible, en este trabajo buscamos que usted y su equipo realice una investigación que termine con una propuesta de proyecto de transformación digital para su país que justifique por qué es necesaria realizarla. Para ello, debe mostrar que los objetivos del proyecto tributan a los objetivos  de desarrollo sostenible de la Agenda 2030 de la ONU \\ \url{https://www.un.org/sustainabledevelopment/es/objetivos-de-desarrollo-sostenible/}.  \\
Instrucciones:
\begin{itemize}
    \item Bajar Formato\footnote{https://www.ieee.org/content/dam/ieee-org/ieee/web/org/conferences/conference-template-a4.docx}. 
    
    \item Mínimo 8 páginas - Ideal 10 páginas - Máximo 12 páginas
    \item Estructuras obligatorias
    \begin{itemize}
        \item \textbf{Título}       
        \item \textbf{Resumen} [10\%]: Un buen resumen debe permitir al lector identificar, en forma rápida y precisa, el contenido básico del trabajo;  no debe tener más de 200 palabras y debe redactarse en pasado (por eso escríbalo al inicio y reescríbalo muchas veces de ser necesario hasta que finalice el trabajo),  escrito en un solo párrafo finalizando con una frase concluyente. No debe aportar información o conclusión que no está presente en el escrito, así como tampoco debe  citar referencias bibliográficas. Debe quedar claro el problema que se investiga y el objetivo del mismo. Dado que este es una investigación para realizar un proyecto debe considerar los siguientes puntos:
        \begin{itemize}
            \item \textbf{Propósito}: Aquí es donde explica ``por qué'' es importante buscar una solución para la problemática. Entregue contexto. Esta es su oportunidad para que los lectores sepan por qué eligió estudiar este tema o problema y su relevancia. 
            \item \textbf{Propuesta}:  Deje que los lectores sepan exactamente lo que se propone realizar para resolver el problema, Cuál será su metodología para verificar que su propuesta aborda las causas raíces del problema. Por ejemplo, ¿realizará entrevistas? ¿Realizará un experimento en el laboratorio? ¿Qué herramientas, métodos, protocolos o conjuntos de datos utilizará para concluir, pivotear, comparar?
            \item \textbf{Originalidad} / valor: esta es su oportunidad de proporcionar a los lectores un análisis del valor de lo que generará esta propuesta.  Es una buena idea preguntar a los colegas si su análisis es equilibrado y justo y, una vez más, es importante no exagerar. 
        \end{itemize}



        \item \textbf{Introducción} [10\%]: debe describir los objetivos de su trabajo, así como describir por qué el tema es importante y cómo contribuye al ámbito local. También debe proporcionar antecedentes tales como estadísticas que permitan llamar la atención en cuanto a la dimensión del problema y explicar qué le hizo decidir investigar este tema. En general, es bueno ser conciso: es importante que la introducción no abrume al resto de su trabajo. \textbf{Esta es su oportunidad de convencer de que el lector siga leyendo}  - resalte la estructura del documento para guiarlo a aquello que sea de su interés.
        
        
        \item \textbf{Estado del arte}[10\%]: -  Es importante responder la siguiente pregunta: ¿Nadie más ha trabajado antes que usted en este tema? Si la respuesta es sí, qué le faltó a ese trabajo que sigue usted en este tema. Para responder a esta pregunta, usted deberá leer o considerar al menos 10 fuentes debidamente referenciadas ( al menos considere un  artículo publicado en revistas académica de corriente principal, o congresos - busque en google scholar de tipo open access).  Esta recopilación debe ser adecuadamente referenciada (Norma APA) - no más de 8 líneas por fuente. Recuerde \textbf{REFERENCIAR} - si usted utiliza el texto de otro autor, y lo copia, es plagio. Si usted utiliza la idea de otro autor, y parafrasea ese texto para hacerlo propio y no lo referencia, usted también comete plagio (use herramientas disponibles tales como https://plagiarismdetector.net/es). No debe utilizar citas textuales (normalmente entre “…”), pues no se trata de la producción de un collage. Evite los sitios web, las fuentes periodísticas o que carezcan de un adecuado respaldo. Evite copiar textos de bots de IA generativa sin validar exhaustivamente la información de manera de contrastarla, corregirla o mejorarla. Finalmente en el último párrafo de esta sección usted podrá concluir respecto de las áreas que aún no son trabajadas suficientemente, o en las que hay espacios de mejora. Indicando la importancia de su propuesta. 


        \item \textbf{Propuesta} [30\%]: Objetivos tributando a objetivos superiores (ODS), Alcance, Limitaciones y Supuestos. Explique a todos la arquitectura propuesta utilizando diagrama de casos de uso, componentes, despliegue, paquetes y procesos describiendo el perfil operacional y el nivel de calidad de servicio a comprometer (utilice el modelo de McCall).  Discusión de Factibilidad de Implementación (indicando tecnologías (apartado para las tecnologías emergentes), frameworks, ambiente). Recuerde limitarse solo a la prueba de concepto que realizaremos en el marco de este curso. Esta propuesta será presentada como muchas otras alternativas - los recursos son escasos - por tanto se escogerán aquellos proyectos que tributen más al País o gobierno, a su misión, visión, objetivos estratégicos, que brinden mejores servicios a sus ciudadanos que les permitan mejorar su calidad de vida. Considera arquitectura empresarial y de software usando C4 model.
        \section{Prueba de Concepto} [30\%]
        Considera una prueba de concepto implementada en una IDE con soporte de IA generativa como Cursor o VSCode con plugins como Copilot. Se considera que el proyecto será desplegado en un ambiente local. Debe tener un README.md con instrucciones para que el profesor pueda instalar y   ejecutar el proyecto en su ambiente local. Debe utilizar un repositorio de github para el proyecto. Debe haber evidencia de que todos los integrantes del equipo han contribuido al repositorio. Debe usar escenarios de uso para explicar como el sistema permite resolver el problema y aportar valor al País.         
        \item \textbf{Conclusiones y trabajo futuro} [10\%]:  Clasifique su proyecto en una de las dimensiones de ciudades inteligentes tales como: economía inteligente, movilidad inteligente, un entorno inteligente, personas inteligentes, vida inteligente y gobernanza inteligente. Indique desafíos tecnológicos, científicos, legales, culturales (del cambio organizacional requerido), grado de novedad respecto al estado del arte especificando los detalles que la hacen diferente (incluso a nivel internacional) - cómo este proyecto aporta al valor agregado del gobierno de un País. 
        \item \textbf{Referencias tipo APA} [- 1 punto máximo]
        \item  \textbf{Otros} (respeta largo, palabras claves relacionadas al paper, ortografía, redacción, palabras claves no adecuadas) [- 1 punto máximo]
        
    \end{itemize}
\end{itemize}
   

