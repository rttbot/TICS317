\chapterimage{ia.png}
\chapter{Ejercicios}

\section*{Pregunta 1}

\begin{flushright}
   \textit{(- puntos)}
\end{flushright}
%\vspace{20px}


\section{Tipo de preguntas o Ejercicios que debemos resolver al finalizar la asignatura} 
\subsection{Ejercicio 1: Identificación de Alcances}
\textbf{Instrucciones:} Clasifica los siguientes escenarios como parte de la arquitectura empresarial (AE) o de la arquitectura de software (AS). Explica tu elección en cada caso.

\begin{enumerate}
    \item Decidir los estándares de interoperabilidad entre las aplicaciones de diferentes departamentos.
    \item Elegir una estructura de base de datos específica para una aplicación de inventarios.
    \item Diseñar una estrategia para cumplir con regulaciones legales internacionales en todas las aplicaciones corporativas.
    \item Seleccionar un patrón de diseño para la implementación de un sistema de notificaciones.
\end{enumerate}

\subsection{Ejercicio 2: Análisis de Roles}
\textbf{Instrucciones:} Lee las siguientes responsabilidades y asigna si corresponden al arquitecto empresarial (AE) o al arquitecto de aplicaciones (AA). Explica por qué.

\begin{enumerate}
    \item Garantizar que todas las aplicaciones dentro de la organización cumplan con los estándares de seguridad definidos.
    \item Optimizar el rendimiento de un servicio específico utilizado por un equipo de desarrollo.
    \item Definir cómo las aplicaciones de marketing y ventas deben integrarse para compartir datos.
    \item Crear diagramas detallados de componentes y flujos de datos de una aplicación.
\end{enumerate}

\subsection{Ejercicio 3: Diseño de Escenarios}
\textbf{Instrucciones:} Dado un escenario, define las actividades que llevarían a cabo tanto el arquitecto empresarial como el arquitecto de software.

\textbf{Escenario:} Una empresa desea lanzar una nueva plataforma de comercio electrónico, integrada con su sistema de gestión de inventarios existente y cumpliendo con normativas de privacidad de datos en todos los mercados donde opera.

\textbf{Arquitecto Empresarial:}
\begin{enumerate}
    \item Enumera tres actividades clave que realizaría para asegurar que el sistema funcione dentro del ecosistema empresarial.
\end{enumerate}

\textbf{Arquitecto de Software:}
\begin{enumerate}
    \item Detalla tres decisiones técnicas que tomaría para diseñar la arquitectura de la aplicación de comercio electrónico.
\end{enumerate}

%\section{Modelos}
\section{Ciclo iterativo de creación de arquitectura de software}
Cuando un ingeniero se enfrenta con un problema complejo, los ingenieros hacen uso de su capacidad de abstracción.  Claramente los problemas simples pueden resolverse directamente. Sino, mapeamos el problema a un modelo abstracto (por ejemplo una ecuación), resolvemos el problema utilizando el modelo, y luego traducimos esa solución al mundo real.


\begin{flushright}
    \textit{Los modelos y abstracciones son útiles cuando queremos explicar o aprender del sistema. Probablemente más útiles que inspeccionar directamente el código fuente. }
\end{flushright}

%%%%%%


\newpage
\section{¿Por qué evaluar una arquitectura}
Cuanto más temprano se encuentre un problema en un proyecto, menor impacto. Una arquitectura es adecuada cuando el sistema resultante de ella cumple los objetivos de calidad con la meta de mitigar los riesgos. 



\newpage
\section{Control - Caso}


\section{Componentes Lógicos: Los Bloques Fundamentales}

Los componentes lógicos son los bloques funcionales fundamentales de un sistema que describen cómo sus piezas encajan entre sí. En la mayoría de los lenguajes de programación, estos componentes se representan a través de la estructura de directorios del repositorio de código fuente.

Para ilustrar los conceptos de esta sección, continuaremos con nuestro ejemplo del sistema de subastas en línea "Adventurous Auctions" y el sistema DART.

\subsection{Arquitectura Lógica vs Física}

Es fundamental comprender la diferencia entre estos dos tipos de arquitectura:

\begin{itemize}
    \item \textbf{Arquitectura Lógica}: Muestra los bloques funcionales del sistema y sus interacciones (acoplamiento). Por ejemplo, en Adventurous Auctions, incluiría componentes como "Gestión de Subastas" y "Registro de Pujadores".
    \item \textbf{Arquitectura Física}: Muestra elementos como servicios, bases de datos, y protocolos. En nuestro ejemplo, incluiría elementos como "Base de Datos de Usuarios" y "Servidor de Streaming".
\end{itemize}

\subsection{Identificación de Componentes Lógicos}

Existen dos enfoques principales para identificar los componentes lógicos iniciales:

\subsubsection{Enfoque de Flujo de Trabajo}

Este enfoque identifica componentes siguiendo el flujo principal del sistema. Para Adventurous Auctions y DART:

\begin{enumerate}
    \item Usuario se registra \(\rightarrow\) \textbf{Componente de Registro}
    \begin{itemize}
        \item Validar información del usuario
        \item Almacenar datos de tarjeta de crédito
        \item Crear perfil de pujador
    \end{itemize}
    
    \item Usuario busca subastas \(\rightarrow\) \textbf{Componente de Búsqueda}
    \begin{itemize}
        \item Listar subastas activas
        \item Filtrar por categoría
        \item Mostrar detalles de viajes
    \end{itemize}
    
    \item Usuario participa en subasta \(\rightarrow\) \textbf{Componente de Pujas}
    \begin{itemize}
        \item Procesar pujas en tiempo real
        \item Validar montos
        \item Notificar al subastador
    \end{itemize}
\end{enumerate}

\subsubsection{Enfoque Actor/Acción}

Este enfoque comienza identificando los actores principales y sus acciones:

\begin{itemize}
    \item \textbf{Pujador Online}:
        \begin{itemize}
            \item Registrarse \(\rightarrow\) \textbf{Gestión de Usuarios}
            \item Ver subastas \(\rightarrow\) \textbf{Catálogo de Subastas}
            \item Realizar pujas \(\rightarrow\) \textbf{Procesador de Pujas}
        \end{itemize}
    \item \textbf{Subastador}:
        \begin{itemize}
            \item Iniciar subasta \(\rightarrow\) \textbf{Control de Subastas}
            \item Gestionar pujas presenciales \(\rightarrow\) \textbf{Integración de Pujas}
            \item Cerrar subasta \(\rightarrow\) \textbf{Procesamiento de Pagos}
        \end{itemize}
\end{itemize}

\subsection{La Trampa de las Entidades}

Es común caer en la "trampa de las entidades" al identificar componentes lógicos. Por ejemplo, en Adventurous Auctions, podríamos estar tentados a crear componentes como:

\begin{itemize}
    \item \textbf{Gestor de Pujas} (¡Mal!)
    \item \textbf{Administrador de Usuarios} (¡Mal!)
    \item \textbf{Controlador de Subastas} (¡Mal!)
\end{itemize}

Estos nombres son demasiado vagos y suelen convertirse en componentes que hacen demasiadas cosas. En su lugar, deberíamos tener:

\begin{itemize}
    \item \textbf{Procesador de Pujas en Tiempo Real}
    \item \textbf{Registro y Autenticación de Usuarios}
    \item \textbf{Coordinador de Flujo de Subasta}
\end{itemize}

\subsection{Acoplamiento de Componentes}

El acoplamiento entre componentes es crucial para la mantenibilidad del sistema. En Adventurous Auctions identificamos dos tipos principales:

\begin{itemize}
    \item \textbf{Acoplamiento Aferente (CA)}: Por ejemplo, varios componentes dependen del "Procesador de Pujas en Tiempo Real":
        \begin{itemize}
            \item Interfaz de Usuario (para mostrar pujas)
            \item Sistema de Notificaciones (para alertar a ganadores)
            \item Procesador de Pagos (para cobrar al ganador)
        \end{itemize}
    
    \item \textbf{Acoplamiento Eferente (CE)}: El "Procesador de Pujas" depende de:
        \begin{itemize}
            \item Validador de Usuarios
            \item Sistema de Almacenamiento de Pujas
            \item Notificador de Eventos
        \end{itemize}
\end{itemize}

Para reducir el acoplamiento, aplicamos la Ley de Demeter, que establece que los componentes deben tener conocimiento limitado de otros componentes. Por ejemplo, el Procesador de Pujas no necesita saber cómo se realiza el pago, solo necesita notificar que una puja ha ganado.