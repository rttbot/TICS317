
\chapter{Caso de Estudio: AD\&D Hotels}
\chapterimage{hotel.png} % Asegúrate de tener esta imagen en la carpeta Pictures

\section{Descripción del Negocio}
AD\&D Hotels es una cadena hotelera de tamaño mediano con aproximadamente 300 hoteles que ha experimentado un crecimiento significativo en los últimos años. La empresa se encuentra en proceso de transformación digital para optimizar sus operaciones y mejorar la experiencia del cliente.

\section{Arquitectura Actual}
\subsection{Componentes del Sistema}
La infraestructura de TI de la empresa está compuesta por varios sistemas críticos:
\begin{itemize}
    \item Sistema de Gestión de Propiedades (PMS)
    \item Sistema de Análisis Comercial
    \item Sistema de Reservas Empresarial
    \item Sistema de Gestión de Canales
    \item Sistema de Precios Hoteleros (componente central)
\end{itemize}

\subsection{Estado Actual}
El sistema actual presenta las siguientes características:
\begin{itemize}
    \item Implementación en la nube mediante estrategia "lift and shift"
    \item Utilización subóptima de recursos cloud
    \item Integración limitada entre componentes
    \item Arquitectura monolítica heredada
\end{itemize}

\section{Desafíos y Oportunidades}
\subsection{Desafíos Técnicos}
\begin{itemize}
    \item Modernización de aplicaciones legacy
    \item Optimización de costos en la nube
    \item Mejora en la integración de sistemas
    \item Escalabilidad limitada
\end{itemize}

\subsection{Oportunidades de Mejora}
\begin{itemize}
    \item Implementación de arquitectura de microservicios
    \item Adopción de prácticas DevOps
    \item Optimización de recursos cloud
    \item Mejora en la experiencia del usuario
\end{itemize}

\section{Arquitectura Propuesta}
\subsection{Visión General}
La nueva arquitectura propone una transición hacia:
\begin{itemize}
    \item Arquitectura basada en microservicios
    \item Implementación de patrones cloud-native
    \item Sistema de mensajería event-driven
    \item Automatización de operaciones
\end{itemize}

\subsection{Componentes Principales}
[Aquí se puede incluir un diagrama de la arquitectura propuesta]

\section{Plan de Implementación}
\subsection{Fases del Proyecto}
\begin{enumerate}
    \item Evaluación y planificación detallada
    \item Proof of Concept (PoC) con componentes seleccionados
    \item Implementación gradual por módulos
    \item Migración de datos y sistemas
    \item Validación y optimización
\end{enumerate}

\subsection{Consideraciones de Seguridad}
\begin{itemize}
    \item Implementación de IAM robusto
    \item Cifrado de datos en reposo y en tránsito
    \item Monitoreo y logging centralizado
    \item Cumplimiento de regulaciones del sector hotelero
\end{itemize}

\section{Conclusiones y Recomendaciones}
[Aquí se pueden incluir las conclusiones y recomendaciones específicas del caso]

\section{Resumen Ejecutivo}
AD\&D Hotels es una cadena hotelera de tamaño mediano con aproximadamente 300 hoteles que ha experimentado un crecimiento significativo en los últimos años. Este documento describe la arquitectura actual y propuesta para su sistema de gestión hotelera.

\section{Descripción General del Sistema}
\subsection{Propósito}
El sistema tiene como objetivo principal gestionar todas las operaciones hoteleras de AD\&D Hotels, incluyendo reservas, gestión de propiedades, análisis comercial y gestión de precios.

\subsection{Alcance}
El sistema abarca los siguientes componentes principales:
\begin{itemize}
    \item Sistema de Gestión de Propiedades (PMS)
    \item Sistema de Análisis Comercial
    \item Sistema de Reservas Empresarial
    \item Sistema de Gestión de Canales
    \item Sistema de Precios Hoteleros (componente central)
\end{itemize}

\section{Arquitectura Actual}
\subsection{Vista General}
La arquitectura actual está basada en un modelo "lift and shift" en la nube, lo que significa que la infraestructura existente se migró sin modificaciones significativas.

\subsection{Componentes Principales}
\subsubsection{Sistema de Gestión de Propiedades}
[Descripción detallada del PMS]

\subsubsection{Sistema de Análisis Comercial}
[Descripción del sistema de análisis]

\subsubsection{Sistema de Reservas}
[Descripción del sistema de reservas]

\section{Desafíos Actuales}
\begin{itemize}
    \item Utilización subóptima de recursos en la nube
    \item Integración limitada entre sistemas
    \item Escalabilidad restringida por la arquitectura monolítica
\end{itemize}

\section{Arquitectura Propuesta}
\subsection{Visión General}
[Descripción de la arquitectura propuesta]

\subsection{Mejoras Planificadas}
\begin{itemize}
    \item Migración a microservicios
    \item Implementación de arquitectura event-driven
    \item Optimización de recursos en la nube
\end{itemize}

\section{Consideraciones de Seguridad}
\subsection{Autenticación y Autorización}
[Detalles de seguridad]

\subsection{Protección de Datos}
[Estrategias de protección de datos]

\section{Plan de Implementación}
\subsection{Fases}
[Descripción de las fases de implementación]

\subsection{Cronograma}
[Cronograma tentativo]

\section{Anexos}
\subsection{Diagramas de Arquitectura}
[Incluir diagramas relevantes]

\subsection{Matriz de Riesgos}
[Detallar riesgos y mitigaciones]

