\documentclass{article}
\usepackage[utf8]{inputenc}
\usepackage{graphicx}
\usepackage{color}
\begin{document}
\begin{center}
    \includegraphics[width=0.5\textwidth]{Pictures/logo.png} \\ % Asegúrate de tener el archivo logo.png en el directorio correcto
    \textbf{FACULTAD DE INGENIERÍA Y CIENCIAS} \\ 
    \textbf{UNIVERSIDAD ADOLFO IBÁÑEZ} \\ 

\end{center}
\title{Control \\ Guía de Estudio para el control de la arquitectura de sistemas}
\author{TICS317 Arquitectura de Sistemas \\ Profesores: Romina Torres, Juan Gana, Eliana Vivas \\ Última actualización: 25 de marzo de 2025}
\date{}
\maketitle



\begin{flushleft}
\textbf{Planteamiento del caso:} 
Una empresa de manufactura de componentes electrónicos de alta precisión para la industria automotriz está implementando tecnologías IoT para mejorar sus procesos. La empresa busca: (1) monitorear en tiempo real la calibración de máquinas de precisión, (2) controlar la calidad de los componentes durante el proceso de producción, (3) gestionar el consumo energético de la planta, y (4) predecir mantenimiento de equipos críticos.
\end{flushleft}

\begin{enumerate}
    \item ¿Qué rol desempeña el arquitecto empresarial en el contexto de la arquitectura empresarial?
    \begin{itemize}
        \item A) Controla la funcionalidad de aplicaciones específicas
        \item B) Diseña el ecosistema dentro del cual las aplicaciones contribuyen a la empresa
        \item C) Desarrolla aplicaciones individuales
        \item D) Supervisa la implementación de software
    \end{itemize}

    \item ¿Cuál es uno de los beneficios de implementar una arquitectura empresarial (AE)?
    \begin{itemize}
        \item A) Aumenta la complejidad de los sistemas de información
        \item B) Facilita la colaboración entre equipos
        \item C) Incrementa los costos operativos
        \item D) Reduce la resiliencia organizacional
    \end{itemize}

    \item ¿Qué lenguaje de modelado estándar se utiliza para la arquitectura empresarial?
    \begin{itemize}
        \item A) UML
        \item B) ArchiMate
        \item C) BPMN
        \item D) SysML
    \end{itemize}

    \item ¿Cuál es uno de los dominios principales de la arquitectura empresarial según TOGAF?
    \begin{itemize}
        \item A) Arquitectura de Seguridad
        \item B) Arquitectura de Negocio
        \item C) Arquitectura de Proyectos
        \item D) Arquitectura de Marketing
    \end{itemize}

    \item ¿Qué principio fundamental de la Ingeniería de Software a Gran Escala (VLSE) se describe cuando se menciona la modularidad?
    \begin{itemize}
        \item A) Complejidad
        \item B) Modularidad
        \item C) Centralización
        \item D) Descentralización
    \end{itemize}

    \item ¿Qué propiedad de calidad se está priorizando cuando se menciona que el sistema debe ser capaz de manejar picos de carga extremos sin comprometer el rendimiento?
    \begin{itemize}
        \item A) Seguridad
        \item B) Escalabilidad
        \item C) Flexibilidad
        \item D) Simplicidad
    \end{itemize}

    \item Ejercicio 2: Análisis de Roles. Lee las siguientes responsabilidades y asigna si corresponden al arquitecto empresarial (AE) o al arquitecto de aplicaciones (AA). Explica por qué.
    \begin{enumerate}
        \item Garantizar que todas las aplicaciones dentro de la organización cumplan con los estándares de seguridad definidos.
        \begin{itemize}
            \item A) Arquitecto Empresarial (AE)
            \item B) Arquitecto de Aplicaciones (AA)
        \end{itemize}
        \item Optimizar el rendimiento de un servicio específico utilizado por un equipo de desarrollo.
        \begin{itemize}
            \item A) Arquitecto Empresarial (AE)
            \item B) Arquitecto de Aplicaciones (AA)
        \end{itemize}
        \item Definir cómo las aplicaciones de marketing y ventas deben integrarse para compartir datos.
        \begin{itemize}
            \item A) Arquitecto Empresarial (AE)
            \item B) Arquitecto de Aplicaciones (AA)
        \end{itemize}
        \item Crear diagramas detallados de componentes y flujos de datos de una aplicación.
        \begin{itemize}
            \item A) Arquitecto Empresarial (AE)
            \item B) Arquitecto de Aplicaciones (AA)
        \end{itemize}
    \end{enumerate}

    \item Instrucciones: Dado un escenario, selecciona las actividades que corresponden al arquitecto empresarial.
    
    Escenario: Una empresa desea lanzar una nueva plataforma de comercio electrónico, integrada con su sistema de gestión de inventarios existente y cumpliendo con normativas de privacidad de datos en todos los mercados donde opera.
    
    \begin{itemize}
        \item A) Asegurar la alineación de la nueva plataforma con los objetivos estratégicos de la empresa.
        \item B) Seleccionar el stack tecnológico adecuado para la plataforma de comercio electrónico.
        \item C) Definir las políticas de privacidad de datos que deben cumplirse.
        \item D) Diseñar la arquitectura de la aplicación para asegurar escalabilidad y rendimiento.
        \item E) Coordinar la integración de la plataforma con el sistema de gestión de inventarios.
    \end{itemize}

\end{enumerate}



    \newpage
    \textbf{Contexto:} Una empresa de manufactura está adoptando tecnologías de IoT para optimizar sus procesos de producción y reducir el desperdicio. La empresa de manufactura se dedica a la producción de componentes electrónicos de alta precisión para la industria automotriz. A continuación se presenta un Balanced Scorecard que refleja los objetivos estratégicos de la empresa.
    
    \begin{table}[h!]
    \centering
    \begin{tabular}{|l|l|l|l|l|}
    \hline
    \textbf{Perspectiva} & \textbf{Objetivo Estratégico} & \textbf{KPI} & \textbf{Valor Actual} & \textbf{Meta a 1 Año / 5 Años} \\ \hline
    Financiera & Reducir costos operativos & Costos operativos totales & \$10M & \$9M / \$7M \\ \hline
    Financiera & Aumentar ROI & ROI de IoT & 5\% & 8\% / 15\% \\ \hline
    Cliente & Mejorar satisfacción del cliente & Índice de satisfacción & 75\% & 80\% / 90\% \\ \hline
    Cliente & Aumentar calidad del producto & Tasa de defectos & 3\% & 2.5\% / 1\% \\ \hline
    Procesos Internos & Optimizar producción & Tiempo de ciclo & 10 horas & 8 horas / 5 horas \\ \hline
    Procesos Internos & Reducir desperdicio & Porcentaje de desperdicio & 10\% & 8\% / 5\% \\ \hline
    Aprendizaje & Mejorar competencias & Horas de capacitación & 20 horas/año & 30 horas/año / 50 horas/año \\ \hline
    Aprendizaje & Fomentar innovación & Nuevas tecnologías & 2 tecnologías & 4 tecnologías / 10 tecnologías \\ \hline
    \end{tabular}
    \caption{Balanced Scorecard de la empresa de manufactura}
    \end{table}
    

\section*{Parte 1: Construcción de la Vista Motivacional}

\begin{enumerate}
    \item \textbf{Identificación de Stakeholders:}
    Describe quiénes son los stakeholders clave en el proyecto de adopción de IoT y cuál es su interés específico en el proyecto.
    
    \color{blue}{\textit{Ejemplo de respuesta: Los stakeholders clave incluyen: (1) Director de Operaciones, interesado en la eficiencia operativa y reducción de costos de la planta, (2) Gerente de Producción, responsable de mantener la calidad de los componentes electrónicos y cumplir con estándares automotrices, (3) Equipo de TI, encargado de implementar la infraestructura IoT y asegurar la integración con sistemas existentes, (4) Clientes de la industria automotriz, que requieren componentes de alta precisión y calidad certificada.}}
    
    \begin{table}[h!]
    \centering
    \begin{tabular}{|c|p{10cm}|}
    \hline
    \textbf{Puntaje} & \textbf{Criterio} \\ \hline
    0 & No identifica stakeholders o identifica stakeholders irrelevantes para el contexto. \\ \hline
    1 & Identifica stakeholders relevantes pero no explica su interés específico en el proyecto IoT. \\ \hline
    2 & Identifica correctamente los stakeholders y explica claramente su interés específico en relación con la manufactura de componentes electrónicos. \\ \hline
    \end{tabular}
    \end{table}

    \item \textbf{Definición de Drivers:}
    Enumera los drivers que motivan la adopción de IoT en la empresa y explica cómo se relacionan con las necesidades específicas de la industria automotriz.
    
    \color{blue}{\textit{Ejemplo de respuesta: Los drivers incluyen: (1) Eficiencia Operativa, para mantener la precisión en la calibración de máquinas y reducir tiempos de inactividad, (2) Reducción de Costos, mediante la optimización del consumo energético y mantenimiento predictivo, (3) Sostenibilidad, para cumplir con regulaciones ambientales de la industria automotriz, (4) Calidad del Producto, para mantener la tasa de defectos por debajo del 3\% requerido por los clientes automotrices.}}
    
    \begin{table}[h!]
    \centering
    \begin{tabular}{|c|p{10cm}|}
    \hline
    \textbf{Puntaje} & \textbf{Criterio} \\ \hline
    0 & No enumera drivers o los drivers no están relacionados con la industria. \\ \hline
    1 & Enumera drivers generales sin vincularlos a las necesidades específicas de la manufactura de componentes electrónicos. \\ \hline
    2 & Enumera drivers relevantes y los vincula claramente con las necesidades específicas de la industria automotriz. \\ \hline
    \end{tabular}
    \end{table}

    \item \textbf{Establecimiento de Objetivos:}
    Define los objetivos estratégicos específicos que la empresa busca alcanzar con la implementación de IoT.
    
    \color{blue}{\textit{Ejemplo de respuesta: Los objetivos incluyen: (1) Reducir la tasa de defectos al 1\% mediante monitoreo en tiempo real de la calibración, (2) Disminuir el consumo energético en un 20\% a través de la gestión inteligente de la planta, (3) Implementar mantenimiento predictivo para reducir el tiempo de inactividad en un 30\%, (4) Alcanzar certificación ISO 9001:2015 para procesos de manufactura con IoT.}}
    
    \begin{table}[h!]
    \centering
    \begin{tabular}{|c|p{10cm}|}
    \hline
    \textbf{Puntaje} & \textbf{Criterio} \\ \hline
    0 & No define objetivos o los objetivos no son medibles. \\ \hline
    1 & Define objetivos generales sin métricas específicas. \\ \hline
    2 & Define objetivos SMART (Específicos, Medibles, Alcanzables, Relevantes y Temporales) alineados con la industria. \\ \hline
    \end{tabular}
    \end{table}

    \item \textbf{Evaluación de Datos del Balance:}
    Describe cómo las evaluaciones (assessments) se relacionan con los objetivos y drivers específicos.
    
    \color{blue}{\textit{Ejemplo de respuesta: Las evaluaciones clave incluyen: (1) Tasa de defectos actual del 3\% que debe reducirse al 1\% para mejorar la calidad del producto, (2) Meta de reducción del tiempo de inactividad en un 30\% para aumentar la eficiencia operativa, (3) Medición del consumo energético actual para establecer metas de reducción que apoyen la sostenibilidad y reducción de costos. Estas evaluaciones están directamente vinculadas con los drivers de Eficiencia Operativa y Sostenibilidad.}}
    
    \begin{table}[h!]
    \centering
    \begin{tabular}{|c|p{10cm}|}
    \hline
    \textbf{Puntaje} & \textbf{Criterio} \\ \hline
    0 & No identifica las evaluaciones específicas o no las relaciona con objetivos/drivers. \\ \hline
    1 & Identifica algunas evaluaciones pero no establece claramente su relación con objetivos y drivers específicos. \\ \hline
    2 & Identifica correctamente las evaluaciones (tasa de defectos, tiempo de inactividad, consumo energético) y las vincula claramente con objetivos y drivers. \\ \hline
    \end{tabular}
    \end{table}

    \item \textbf{Determinación de Outputs:}
    Identifica los outputs técnicos necesarios para alcanzar los objetivos y cómo se relacionan con las evaluaciones.
    
    \color{blue}{\textit{Ejemplo de respuesta: Los outputs incluyen un Plan de Implementación de IoT y un Informe de Evaluación de Impacto, que aseguran el cumplimiento de objetivos estratégicos que les permita lograr consumo energético menor en un 20\%, tiempo de inactividad en un 30\%, Tasa de defectos al 1\%, certificación de manufactura}}
    
    \begin{table}[h!]
    \centering
    \begin{tabular}{|c|p{10cm}|}
    \hline
    \textbf{Puntaje} & \textbf{Criterio} \\ \hline
    0 & No identifica los outputs técnicos o no los relaciona con evaluaciones. \\ \hline
    1 & Identifica algunos outputs pero no establece su relación con evaluaciones específicas. \\ \hline
    2 & Identifica correctamente los cuatro outputs técnicos (monitoreo TR, gestión inteligente, gestión energética, predicción mantenimiento) y los vincula con evaluaciones específicas. \\ \hline
    \end{tabular}
    \end{table}

    \item \textbf{Especificación de Requerimientos:}
    Enumera los requerimientos técnicos necesarios para implementar los outputs identificados y explica cómo soportan las evaluaciones.
    
    \color{blue}{\textit{Ejemplo de respuesta: Los requerimientos técnicos incluyen: (1) sistema de monitoreo en tiempo real de calibración - Sensores de calibración y sistema de monitoreo en tiempo real para control de calidad, (2) sistema de gestión inteligente de planta - Infraestructura IoT y red de sensores para la gestión inteligente de la planta, (3) sistema inteligente de gestión del consumo energético - Medidores inteligentes y sistema de análisis para gestión del consumo energético, (4) sistema de predicción de mantenimiento de equipos críticos - Sensores de condición de equipos y algoritmos predictivos para mantenimiento. Cada requerimiento está diseñado para soportar un output específico y permite alcanzar las metas establecidas en las evaluaciones.}}
    
    \begin{table}[h!]
    \centering
    \begin{tabular}{|c|p{10cm}|}
    \hline
    \textbf{Puntaje} & \textbf{Criterio} \\ \hline
    0 & No especifica requerimientos técnicos o no los relaciona con outputs. \\ \hline
    1 & Especifica algunos requerimientos pero no establece claramente su relación con outputs y evaluaciones. \\ \hline
    2 & Especifica correctamente los requerimientos técnicos y los vincula claramente con outputs y evaluaciones específicas. \\ \hline
    \end{tabular}
    \end{table}

    \item \textbf{Desarrollo de la Vista Motivacional:}
    Dibuja la vista motivacional basada en los elementos identificados anteriormente.
    
    \color{blue}{\textit{Ejemplo de respuesta: Un diagrama que muestra la relación entre stakeholders, drivers, objetivos, assessments, outputs y requerimientos.}}
    
    \begin{table}[h!]
    \centering
    \begin{tabular}{|c|p{10cm}|}
    \hline
    \textbf{Puntaje} & \textbf{Criterio} \\ \hline
    0 & No presenta un diagrama o es incorrecto. \\ \hline
    1 & Presenta un diagrama pero con errores o incompleto. \\ \hline
    2 & Presenta un diagrama correcto y completo. \\ \hline
    \end{tabular}
    \end{table}

\end{enumerate}

\section*{Parte 2: Construcción de la Vista en Capas (Layered View)}

\begin{enumerate}
    \item \textbf{Definición de la Capa de Negocio:}
    Describe los procesos de negocio clave que soportan los objetivos estratégicos de la empresa de manufactura de componentes electrónicos.
    
    \color{blue}{\textit{Ejemplo de respuesta: Los procesos de negocio clave incluyen: (1) Proceso de Calibración de Máquinas de Precisión, que asegura la calidad en la manufactura de componentes, (2) Proceso de Control de Calidad en Línea, que mantiene la tasa de defectos por debajo del 3\%, (3) Proceso de Gestión Energética, que optimiza el consumo de recursos, (4) Proceso de Mantenimiento Predictivo, que reduce tiempos de inactividad. Estos procesos están directamente alineados con los drivers de Eficiencia Operativa y Calidad del Producto.}}
    
    \begin{table}[h!]
    \centering
    \begin{tabular}{|c|p{10cm}|}
    \hline
    \textbf{Puntaje} & \textbf{Criterio} \\ \hline
    0 & No describe procesos de negocio relevantes para la manufactura de componentes electrónicos. \\ \hline
    1 & Describe procesos generales sin vincularlos a los objetivos específicos de calidad y eficiencia. \\ \hline
    2 & Describe correctamente los procesos clave y los vincula con los objetivos de reducción de defectos, eficiencia energética y mantenimiento. \\ \hline
    \end{tabular}
    \end{table}

    \item \textbf{Identificación de Servicios de Negocio:}
    Enumera los servicios de negocio que se ofrecen a los clientes de la industria automotriz y cómo están soportados por los procesos de negocio.
    
    \color{blue}{\textit{Ejemplo de respuesta: Los servicios clave incluyen: (1) Producción de Componentes de Alta Precisión con tasa de defectos menor al 1\%, soportado por los procesos de calibración y control de calidad, (2) Manufactura Sostenible y Eficiente, respaldada por la gestión energética inteligente, (3) Garantía de Continuidad Operativa, asegurada por el mantenimiento predictivo, (4) Certificación de Calidad Automotriz, sustentada por todos los procesos anteriores.}}
    
    \begin{table}[h!]
    \centering
    \begin{tabular}{|c|p{10cm}|}
    \hline
    \textbf{Puntaje} & \textbf{Criterio} \\ \hline
    0 & No enumera servicios relevantes para la industria automotriz. \\ \hline
    1 & Enumera servicios generales sin vincularlos a los procesos específicos de manufactura. \\ \hline
    2 & Enumera correctamente los servicios y los vincula con los procesos de manufactura de precisión. \\ \hline
    \end{tabular}
    \end{table}

    \item \textbf{Desarrollo de la Capa de Aplicación:}
    Identifica las aplicaciones IoT específicas que soportan los procesos de manufactura y describe su función.
    
    \color{blue}{\textit{Ejemplo de respuesta: Las aplicaciones clave incluyen: (1) Sistema de Monitoreo en Tiempo Real (RTMS) para control de calibración, (2) Sistema de Gestión de Calidad (QMS) integrado con sensores IoT, (3) Sistema de Gestión Energética Inteligente (IEMS) para optimización de consumo, (4) Sistema de Mantenimiento Predictivo (PMS) basado en análisis de datos IoT. Cada aplicación está integrada con sensores específicos y proporciona dashboards en tiempo real.}}
    
    \begin{table}[h!]
    \centering
    \begin{tabular}{|c|p{10cm}|}
    \hline
    \textbf{Puntaje} & \textbf{Criterio} \\ \hline
    0 & No identifica aplicaciones IoT relevantes para manufactura. \\ \hline
    1 & Identifica aplicaciones generales sin especificar su función en el contexto IoT. \\ \hline
    2 & Identifica correctamente las aplicaciones IoT y describe su función específica en la manufactura. \\ \hline
    \end{tabular}
    \end{table}

    \item \textbf{Integración de Aplicaciones:}
    Explica cómo las aplicaciones IoT están integradas para soportar la manufactura de precisión.
    
    \color{blue}{\textit{Ejemplo de respuesta: La integración se realiza mediante: (1) Bus de Servicios Empresariales (ESB) que conecta todos los sistemas IoT, (2) APIs RESTful para intercambio de datos en tiempo real entre sistemas, (3) Base de datos Time-Series para almacenamiento de datos de sensores, (4) Sistema de Mensajería para alertas y notificaciones entre sistemas. Esta integración permite el flujo continuo de datos desde los sensores hasta los dashboards de control.}}
    
    \begin{table}[h!]
    \centering
    \begin{tabular}{|c|p{10cm}|}
    \hline
    \textbf{Puntaje} & \textbf{Criterio} \\ \hline
    0 & No explica la integración de sistemas IoT. \\ \hline
    1 & Explica integración general sin especificar mecanismos para datos en tiempo real. \\ \hline
    2 & Explica correctamente la integración de sistemas IoT y el flujo de datos en tiempo real. \\ \hline
    \end{tabular}
    \end{table}

    \item \textbf{Definición de la Capa de Tecnología:}
    Describe la infraestructura tecnológica que soporta las aplicaciones IoT en la planta de manufactura.
    
    \color{blue}{\textit{Ejemplo de respuesta: La infraestructura incluye: (1) Red de sensores industriales IoT conectados mediante protocolos industriales (MQTT, OPC-UA), (2) Gateway IoT para procesamiento en el borde y filtrado de datos, (3) Servidores en la nube para análisis avanzado y almacenamiento, (4) Red industrial segura con segregación entre IT/OT. Esta infraestructura garantiza la recolección y procesamiento confiable de datos de manufactura.}}
    
    \begin{table}[h!]
    \centering
    \begin{tabular}{|c|p{10cm}|}
    \hline
    \textbf{Puntaje} & \textbf{Criterio} \\ \hline
    0 & No describe infraestructura relevante para IoT industrial. \\ \hline
    1 & Describe infraestructura general sin especificaciones para manufactura. \\ \hline
    2 & Describe correctamente la infraestructura IoT específica para manufactura. \\ \hline
    \end{tabular}
    \end{table}

    \item \textbf{Servicios de Infraestructura:}
    Enumera los servicios de infraestructura necesarios para el funcionamiento de las aplicaciones IoT en manufactura.
    
    \color{blue}{\textit{Ejemplo de respuesta: Los servicios incluyen: (1) Servicio de Conectividad Industrial para sensores IoT, (2) Servicio de Procesamiento en el Borde para datos en tiempo real, (3) Servicio de Almacenamiento y Análisis en la Nube, (4) Servicios de Seguridad Industrial incluyendo firewalls industriales y sistemas de detección de intrusiones (IDS). Estos servicios aseguran la operación continua y segura de la infraestructura IoT.}}
    
    \begin{table}[h!]
    \centering
    \begin{tabular}{|c|p{10cm}|}
    \hline
    \textbf{Puntaje} & \textbf{Criterio} \\ \hline
    0 & No enumera servicios relevantes para IoT industrial. \\ \hline
    1 & Enumera servicios generales sin especificaciones para manufactura. \\ \hline
    2 & Enumera correctamente los servicios específicos para IoT en manufactura. \\ \hline
    \end{tabular}
    \end{table}

    \item \textbf{Evaluación de la Seguridad:}
    Describe las medidas de seguridad implementadas para proteger los sistemas IoT en el entorno de manufactura.
    
    \color{blue}{\textit{Ejemplo de respuesta: Las medidas de seguridad incluyen: (1) Segmentación de red IT/OT mediante firewalls industriales, (2) Autenticación y autorización basada en roles para acceso a sistemas IoT, (3) Encriptación de datos en tránsito y en reposo para información de producción, (4) Monitoreo continuo de seguridad mediante SIEM industrial. Estas medidas protegen tanto la propiedad intelectual como la continuidad operativa.}}
    
    \begin{table}[h!]
    \centering
    \begin{tabular}{|c|p{10cm}|}
    \hline
    \textbf{Puntaje} & \textbf{Criterio} \\ \hline
    0 & No describe medidas de seguridad relevantes para IoT industrial. \\ \hline
    1 & Describe medidas generales sin considerar el contexto de manufactura. \\ \hline
    2 & Describe correctamente las medidas de seguridad específicas para IoT en manufactura. \\ \hline
    \end{tabular}
    \end{table}

    \item \textbf{Desarrollo de la Vista en Capas:}
    Dibuja la vista en capas mostrando cómo las capas de negocio, aplicación y tecnología soportan la manufactura de componentes electrónicos con IoT.
    
    \color{blue}{\textit{Ejemplo de respuesta: Un diagrama que muestra la interconexión entre: (1) Capa de Negocio con los procesos de manufactura, (2) Capa de Aplicación con los sistemas IoT, (3) Capa de Tecnología con la infraestructura de sensores y redes industriales. El diagrama debe mostrar cómo estas capas soportan los objetivos de calidad y eficiencia.}}
    
    \begin{table}[h!]
    \centering
    \begin{tabular}{|c|p{10cm}|}
    \hline
    \textbf{Puntaje} & \textbf{Criterio} \\ \hline
    0 & No presenta un diagrama de capas para manufactura IoT. \\ \hline
    1 & Presenta un diagrama incompleto o sin conexiones claras entre capas. \\ \hline
    2 & Presenta un diagrama completo mostrando claramente cómo las capas soportan la manufactura IoT. \\ \hline
    \end{tabular}
    \end{table}

\end{enumerate}

\end{document} 